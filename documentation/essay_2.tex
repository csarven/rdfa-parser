%%%%%%%%%%%%%%%%%%%%%%%%%%%%%%%%%%%%%%%%%
% Thin Sectioned Essay
% LaTeX Template
% Version 1.0 (3/8/13)
%
% This template has been downloaded from:
% http://www.LaTeXTemplates.com
%
% Original Author:
% Nicolas Diaz (nsdiaz@uc.cl) with extensive modifications by:
% Vel (vel@latextemplates.com)
%
% License:
% CC BY-NC-SA 3.0 (http://creativecommons.org/licenses/by-nc-sa/3.0/)
%
%%%%%%%%%%%%%%%%%%%%%%%%%%%%%%%%%%%%%%%%%

%----------------------------------------------------------------------------------------
%	PACKAGES AND OTHER DOCUMENT CONFIGURATIONS
%----------------------------------------------------------------------------------------

\documentclass[hidelinks, a4paper, 11pt]{article} % Font size (can be 10pt, 11pt or 12pt) and paper size (remove a4paper for US letter paper)

\usepackage[protrusion=true,expansion=true]{microtype} % Better typography
\usepackage{graphicx} % Required for including pictures
\usepackage{wrapfig} % Allows in-line images

\usepackage{hyperref}

\usepackage{mathpazo} % Use the Palatino font
\usepackage[T1]{fontenc} % Required for accented characters
\linespread{1.05} % Change line spacing here, Palatino benefits from a slight increase by default

\makeatletter
\renewcommand\@biblabel[1]{\textbf{#1.}} % Change the square brackets for each bibliography item from '[1]' to '1.'
\renewcommand{\@listI}{\itemsep=0pt} % Reduce the space between items in the itemize and enumerate environments and the bibliography

\renewcommand{\maketitle}{ % Customize the title - do not edit title and author name here, see the TITLE block below
\begin{flushright} % Right align
{\LARGE\@title} % Increase the font size of the title

\vspace{50pt} % Some vertical space between the title and author name

{\large\@author} % Author name
\\\@date % Date

\vspace{40pt} % Some vertical space between the author block and abstract
\end{flushright}
}


%----------------------------------------------------------------------------------------
%	TITLE
%----------------------------------------------------------------------------------------

\title{\textbf{RDFa crawler}\\ % Title
Webcrawler including RDFa parsing} % Subtitle

\author{\textsc{Stefan Achm\"uller, Roland Gritzer, Mathias Gschwandtner} % Author
\\{\textit{Universit\"at Innsbruck}}} % Institution

\date{\today} % Date

%----------------------------------------------------------------------------------------

\begin{document}
\maketitle % Print the title section


%----------------------------------------------------------------------------------------
%	ABSTRACT, KEYWORDS AND TABLE OF CONTENTS
%----------------------------------------------------------------------------------------

%\renewcommand{\abstractname}{Summary} % Uncomment to change the name of the abstract to something else

\begin{abstract}
Enabling to scrap the world wide web for RDFa data. Give a URL to start from, furthermore black- and whitelisting is possible. DRFa data, integrated within html documents, is generated by applying rdfa.info sequence. RDFa data is annotated by the RDF n-triple format.
\end{abstract}

\hspace*{3,6mm}\textit{Keywords:} webcrawler, parser, rdfa % Keywords

\vspace{30pt} % Some vertical space between the abstract and first section


\tableofcontents
\newpage

% GRADING ASPECT
% Result of written Report:
% 	2-6 pages
% 	Problem definition (what and why is it relevant?)
% 	Methodology (how did we solve?)
% 	Outcome (result of our work incl. Readme.md)

% Inline image example
% \begin{wrapfigure}{l}{0.4\textwidth} 
% \begin{center}
% \includegraphics[width=0.38\textwidth]{fish.png}
% \end{center}
% \caption{Fish}
% \end{wrapfigure}


%----------------------------------------------------------------------------------------
%	ESSAY BODY
%----------------------------------------------------------------------------------------

\section{Problem Definition}

Need to get information from web - clientside info building with rdfa, webcrawler to automate info gathering.

%------------------------------------------------

\section{Methodology}

Do research: javascript, node, schema.org, etc.
Split work into parser and crawler.
Set up front end (testing and presentation).
Start implement.
Put together and write documentation.

%------------------------------------------------

\section{Outcome}

1 package to download via node npm.
2 separate functionalities within package: crawler (without parser), and parser (without crawler).


%----------------------------------------------------------------------------------------
%	BIBLIOGRAPHY
%----------------------------------------------------------------------------------------

\bibliographystyle{unsrt}

\bibliography{sample}	% use sample.bib from same absolute path (location)

%----------------------------------------------------------------------------------------

\end{document}
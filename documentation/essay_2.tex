%%%%%%%%%%%%%%%%%%%%%%%%%%%%%%%%%%%%%%%%%
% Thin Sectioned Essay
% LaTeX Template
% Version 1.0 (3/8/13)
%
% This template has been downloaded from:
% http://www.LaTeXTemplates.com
%
% Original Author:
% Nicolas Diaz (nsdiaz@uc.cl) with extensive modifications by:
% Vel (vel@latextemplates.com)
%
% License:
% CC BY-NC-SA 3.0 (http://creativecommons.org/licenses/by-nc-sa/3.0/)
%
%%%%%%%%%%%%%%%%%%%%%%%%%%%%%%%%%%%%%%%%%

%----------------------------------------------------------------------------------------
%	PACKAGES AND OTHER DOCUMENT CONFIGURATIONS
%----------------------------------------------------------------------------------------

\documentclass[hidelinks, a4paper, 11pt]{article} % Font size (can be 10pt, 11pt or 12pt) and paper size (remove a4paper for US letter paper)

\usepackage[protrusion=true,expansion=true]{microtype} % Better typography
\usepackage{graphicx} % Required for including pictures
\usepackage{wrapfig} % Allows in-line images

\usepackage{hyperref} % use links for ToC, etc.


\usepackage{mathpazo} % Use the Palatino font
\usepackage[T1]{fontenc} % Required for accented characters
\linespread{1.05} % Change line spacing here, Palatino benefits from a slight increase by default

\makeatletter
\renewcommand\@biblabel[1]{\textbf{#1.}} % Change the square brackets for each bibliography item from '[1]' to '1.'
\renewcommand{\@listI}{\itemsep=0pt} % Reduce the space between items in the itemize and enumerate environments and the bibliography

\renewcommand{\maketitle}{ % Customize the title - do not edit title and author name here, see the TITLE block below
\begin{flushright} % Right align
{\LARGE\@title} % Increase the font size of the title

\vspace{50pt} % Some vertical space between the title and author name

{\large\@author} % Author name
\\\@date % Date

\vspace{40pt} % Some vertical space between the author block and abstract
\end{flushright}
}


%----------------------------------------------------------------------------------------
%	TITLE
%----------------------------------------------------------------------------------------

\title{\textbf{RDFa Crawler}\\ % Title
Webcrawler inklusive RDFa parsing funktion} % Subtitle

\author{\textsc{Stefan Achm\"uller, Roland Gritzer, Mathias Gschwandtner} % Author
\\{\textit{Universit\"at Innsbruck}}} % Institution

\date{\today} % Date

%----------------------------------------------------------------------------------------

\begin{document}
\maketitle % Print the title section


%----------------------------------------------------------------------------------------
%	ABSTRACT, KEYWORDS AND TABLE OF CONTENTS
%----------------------------------------------------------------------------------------

\renewcommand{\abstractname}{Zusammenfassung} % Uncomment to change the name of the abstract to something else

\begin{abstract}
Enabling to scrap the world wide web for RDFa data. Give a URL to start from, furthermore black- and whitelisting is possible. DRFa data, integrated within html documents, is generated by applying rdfa.info sequence. RDFa data is annotated by the RDF n-triple format.
\end{abstract}

\hspace*{3,6mm}\textit{Schl\"usselw\"orter:} webcrawler, parser, rdfa % Keywords

\vspace{30pt} % Some vertical space between the abstract and first section

\renewcommand{\contentsname}{Inhaltsangabe}

\tableofcontents
\newpage

% GRADING ASPECT
% Result of written Report:
% 	2-6 pages
% 	Problem definition (what and why is it relevant?)
% 	Methodology (how did we solve?)
% 	Outcome (result of our work incl. Readme.md)

% Inline image example
% \begin{wrapfigure}{l}{0.4\textwidth} 
% \begin{center}
% \includegraphics[width=0.38\textwidth]{fish.png}
% \end{center}
% \caption{Fish}
% \end{wrapfigure}


%----------------------------------------------------------------------------------------
%	ESSAY BODY
%----------------------------------------------------------------------------------------

\section{Problem Definition}

Mit der historischen Entwicklung des Internets wurden immer wieder Techniken eingef\"uhrt, die den Datenaustausch zwichen Rechnern vereinfachen. W\"ahrend im klassischen Internet, auch Web 1.0 (URL, HTTP, HTML, etc.) genannt, der Fokus auf dem Aufbau und Transport der Daten liegt, betrachtet der Ansatz "Semantic Web" m\"ogliche Interpretationen der Daten. Dies wird auch Web 3.0 genannt und ermöglicht eine vereinfachte Abarbeitung von Aufgaben, basierend auf Internetdaten. So kann zum Beispiel unterschieden werden, ob das Wort "Bremen" auf einer bestimmten Webseite sich entweder auf eine deutsche Stadt, einen Familiennamen oder einem sonstigen Namen bezieht. Die Kerneigenschaft des "Semantic Web" stellt die Universalit\"at der Relationten dar. Dies bedeutet, dass prinzipiell alle Informationsobjekte miteinander verkn\"upft werden können um Wissen zu repr\"asentieren \cite{berners2001semantic}. \linebreak 

Um Daten mit diesen Metainformationen anzureichern, wurden diverse Annotationstypen eingef\"uhrt. Neben JSON-LD und Microdata, bietet sich hierfür das "Resource Description Frameworks" (RDF) an. Ziel dieser Arbeit ist die Erstellung einer Programmierbibliothek zum automatischen extrahieren von semantischen Informationen, annotiert mittels RDFa (W3C Standard Annotation f\"ur RDF), welche in HTML Webseiten eingebettet sind \cite{halb2008building}.

\subsection{RDFa Information}

RDFa lite und die sequenzen \cite{adida2008rdfa}.


\subsection{Webcrawler}

Webcrawler - finden was man sucht \cite{pinkerton2000webcrawler}.



%------------------------------------------------

\section{Methodologie}

Do research: javascript, node, schema.org, etc.
Split work into parser and crawler.
Set up front end (testing and presentation).
Start implement.
Put together and write documentation.

\subsection{NodeJS und npm}

dfsflkj

\subsection{Modul 1: RDFa Parser}

dsflkj

\subsection{Modul 2: Webcrawler}

dklfja

\subsection{Zusammenf\"uhrung}

dsfjd



%------------------------------------------------

\section{Ergebnis}

1 package to download via node npm.
2 separate functionalities within package: crawler (without parser), and parser (without crawler).

\subsection{Beispielanwendung}

dasflkj

%----------------------------------------------------------------------------------------
%	BIBLIOGRAPHY
%----------------------------------------------------------------------------------------

\renewcommand{\refname}{Referenzen}

\newpage
\bibliography{sample}	% use sample.bib from same absolute path (location)

\bibliographystyle{unsrt}

%----------------------------------------------------------------------------------------

\end{document}